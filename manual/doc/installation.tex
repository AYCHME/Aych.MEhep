\chapter{Installation}

\section{Installing the Dependencies}

Installing EOS from source will require the following software:
\begin{description}
    \item[\package{g++}] the GNU C++ compiler, in version 4.8.1 or higher,
    \item[\package{autoconf}] the GNU tool for creating configure scripts, in version 2.69 or higher,
    \item[\package{automake}] the GNU tool for creating makefiles, in version 1.14.1 or higher,
    \item[\package{libtool}] the GNU tool for generic library support, in version 2.4.2 or higher.
\end{description}
%
Building and using the core libraries requires in addition the following software:
\begin{description}
    \item[\package{GSL}] the GNU Scientific Library \cite{GSL}, in version 1.16 or higher,
    \item[\package{HDF5}] the Hierarchical Data Format v5 library \cite{HDF5}, in version 1.8.11 or higher,
    \item[\package{Minuit2}] the physics analysis tool for function minimization, in version 5.28.00 or higher.
\end{description}
Except for \package{Minuit2}, we recommend the installation of the above packages using your system's
software management system. For \package{Minuit2}, we recommend to install from source, and to disable
the automatic support for OpenMP. The installation can be done using:
\begin{commandline}
mkdir /tmp/Minuit2
pushd /tmp/Minuit2
wget http://www.cern.ch/mathlibs/sw/5_28_00/Minuit2/Minuit2-5.28.00.tar.gz
tar zxf Minuit2-5.28.00.tar.gz
pushd Minuit2-5.28.00
./configure --prefix=/opt/pkgs/Minuit2-5.28.00 --disable-openmp
make all
sudo make install
popd
popd
rm -R /tmp/Minuit2
\end{commandline}

If you intend to use the \gls{PMC} sampling algorithm, you will need to install
\begin{description}
    \item[libpmc] a free implementation of said algorithm \cite{libpmc}, in version 1.01 or higher,
\end{description}
in addition to the core library dependencies. The \package{libpmc} package will -- in all likelihood -- not
be installable via your system's software management system. In addition, EOS requires some
modifications to libpmc's source code, in order to make it compatible with C++. We suggest
the following commands to install it:
\begin{commandline}
mkdir /tmp/libpmc
pushd /tmp/libpmc
wget http://www2.iap.fr/users/kilbinge/CosmoPMC/pmclib_v1.01.tar.gz
tar zxf pmclib_v1.01.tar.gz
pushd pmclib_v1.01
./waf configure --m64 --prefix=/opt/pkgs/pmclib-1.01
./waf
sudo ./waf install
sudo find /opt/pkgs/pmclib-1.01/include -name "*.h" \
    -exec sed -i \
    -e 's/#include "errorlist\.h"/#include <pmctools\/errorlist.h>/' \
    -e 's/#include "io\.h"/#include <pmctools\/io.h>/' \
    -e 's/#include "mvdens\.h"/#include <pmctools\/mvdens.h>/' \
    -e 's/#include "maths\.h"/#include <pmctools\/maths.h>/' \
    -e 's/#include "maths_base\.h"/#include <pmctools\/maths_base.h>/' \
    {} \;
sudo sed -i \
    -e 's/^double fmin(double/\/\/&/' \
    -e 's/^double fmax(double/\/\/&/' \
    /opt/pkgs/pmclib-1.01/include/pmctools/maths.h
popd
popd
rm -R /tmp/pmclib
\end{commandline}

\section{Installing EOS}

The most recent version of EOS is contained in the public GIT \cite{GIT}
repository at \url{http://github.com/eos/eos}.  In order to download it for the
first time, create a new local clone of said repository via:
%
\begin{commandline}
git clone \
    -o eos \
    -b master \
    https://github.com/eos/eos.git \
    eos
\end{commandline}


As a first step, you need to create all the necessary build scripts via:
%
\begin{commandline}
cd eos
./autogen.bash
\end{commandline}
%
Next, you configure the build scripts using:
%
\begin{commandline}
./configure \
    --prefix=/opt/pkgs/eos \
    --with-minuit2=/opt/pkgs/Minuit2-5.28.00 \
    --with-pmc=no
    # --with-pmc=/opt/pkgs/pmclib-1.01
\end{commandline}
%
Note that the last line is optional, and should replace the second to last line only
if you intend to use the \gls{PMC} sampling algorithm. If the \texttt{configure} script
finds any problems with your system, it will complain loudly.\\

After successful configuration, you can build and install EOS using:
%
\begin{commandline}
make all
sudo make install
\end{commandline}
%
In order to be able to use the EOS clients from the command line, you will need to
\begin{commandline}
export PATH+=":/opt/pkgs/eos/bin"
\end{commandline}
to you \texttt{.bashrc} file. In order to build you own programs that use the EOS libraries,
add
\begin{commandline}
CXXFLAGS+=" -I/opt/pkgs/eos/include"
LDFLAGS+=" -L/opt/pkgs/eos/lib"
\end{commandline}
to your makefile.\\

Moreover, we urgently recommend to also run the complete test suite upon installation,
using:
%
\begin{commandline}
make check
\end{commandline}
%
within the source directory. Please contact the authors in the case that any test failures should occur.
