\chapter{Installation}

The following instructions explain how to install \EOS from source on a 
Linux-based operating system, such as Debian
or Ubuntu, \footnote{%
    Other flavors of Linux will work as well, however, note that
    we will exclusively use package names as they appear in the Debian/Ubuntu
    \text{apt} databases.
}
or MacOS X.

\section{Installing the dependencies}

\subsection{Overview}
\label{sec:inst:depend:overview}

The dependencies can be roughly categorized as either system software, Python-related
software, or scientific software.\\


\EOS requires the following system software:
\begin{description}[labelwidth=.15\textwidth]
    \item[\package{g++}] the GNU C++ compiler, in version 4.8.1 or higher;
    \item[\package{autoconf}] the GNU tool for creating configure scripts, in version 2.69 or higher;
    \item[\package{automake}] the GNU tool for creating makefiles, in version 1.14.1 or higher;
    \item[\package{libtool}] the GNU tool for generic library support, in version 2.4.2 or higher;
    \item[\package{pkg-config}] the freedesktop.org library helper, in version 0.28 or higher;
    \item[\package{BOOST}] the BOOST C++ libraries (sub-libraries \package{boost-filesystem} and
            \package{boost-system});
    \item[\package{yaml-cpp}] a C++ YAML parser and interpreter library.
\end{description}
The Phython \cite{Python} interface to \EOS requires the additional software:
\begin{description}[labelwidth=.15\textwidth]
    \item[\package{python3}] python3 interpreter in version 3.5 or higher, and required header files;
    \item[\package{BOOST}] the BOOST C++ library \package{boost-python} for interfacing Python and C++;
    \item[\package{h5py}] the Python interface to HDF5;
    \item[\package{matplotlib}] the Python plotting library;
    \item[\package{scipy}] the Python scientific library;
    \item[\package{PyYAML}] the Python YAML parser and emitter library.
\end{description}
We recommend you install the above packages via your system's software management system.\\


\EOS requires the following scientific software:
\begin{description}[labelwidth=.15\textwidth]
    \item[\package{GSL}] the GNU Scientific Library \cite{GSL}, in version 1.16 or higher;
    \item[\package{HDF5}] the \gls{HDF5} \cite{HDF5}, in version 1.8.11 or higher;
    \item[\package{FFTW3}] the C subroutine library for computing the discrete Fourier transform;
    \item[\package{Minuit2}] the physics analysis tool for function minimization, in version 5.28.00 or higher.
\end{description}
The optional (and highly recommended) \gls{PMC} sampling algorithm requires:
\begin{description}[labelwidth=.15\textwidth]
    \item[\package{pmclib}] a free implementation of said algorithm \cite{libpmc}, in version 1.01 or higher.
\end{description}

We recommend you install the above packages via your system's software
management system, except for \package{pmclib} and \package{Minuit2} in particular
cases discussed below.

\subsection{Linux: Installing the dependencies with \package{APT} (Debian, Ubuntu)}

You can install the prerequisite software with the \package{apt} package
management system on any Debian or Ubuntu system as follows:
\begin{commandline}
# for the 'System Software'
apt-get install g++ autoconf automake libtool pkg-config libboost-filesystem-dev libboost-system-dev libyaml-cpp-dev
# for the 'Python Software'
apt-get install python3-dev libboost-python-dev python3-h5py python3-matplotlib python3-scipy python3-yaml
# for the 'Scientific Software'
apt-get install libgsl0-dev libhdf5-serial-dev libfftw3-dev
\end{commandline}
Do not install \package{Minuit2} via \package{APT}, since there is presently a
bug in the Debian/Ubuntu packages for \package{Minuit2}, which prevents
linking.\\


There are pre-built binary package files for
\package{pmclib} and \package{Minuit2} available for the Ubuntu
long-term-support varieties Xenial and Bionic via the Packagecloud web service.
To use the EOS third-party repository, create a new file \filename{eos.list} within
the directory \directory{/etc/apt/sources.list.d} with the following contents:
\begin{file}
deb https://packagecloud.io/eos/eos/ubuntu/ DIST main
deb-src https://packagecloud.io/eos/eos/ubuntu/ DIST main
\end{file}
where you must replace \texttt{DIST} with either \texttt{xenial} or
\texttt{bionic}, depending on your version of Ubuntu. Add our repository's
GPG key by running
\begin{commandline}
curl -L "https://packagecloud.io/eos/eos/gpgkey" 2> /dev/null | apt-key add - &>/dev/null
\end{commandline}
You can then install the binary packages through
\begin{commandline}
apt-get update
apt-get install minuit2 libpmc
\end{commandline}

You can then proceed with the EOS installation as documented in \refsec{inst:EOS}.

\subsection{Linux: Installing \package{Minuit2} and \package{pmclib} from source}

To install \package{Minuit2} from source, you need to disable the automatic support
for OpenMP. Execute the following commands:
\begin{commandline}
mkdir /tmp/Minuit2
pushd /tmp/Minuit2
wget http://www.cern.ch/mathlibs/sw/5_28_00/Minuit2/Minuit2-5.28.00.tar.gz
tar zxf Minuit2-5.28.00.tar.gz
pushd Minuit2-5.28.00
./configure --prefix=/opt/pkgs/Minuit2-5.28.00 --disable-openmp
make all
sudo make install
popd
popd
rm -R /tmp/Minuit2
\end{commandline}

Installation of \package{pmclib} for \EOS's purposes requires some modifications
to the source code to make it compatible with C++. Execute the following
commands:
\begin{commandline}
mkdir /tmp/libpmc
pushd /tmp/libpmc
wget http://www2.iap.fr/users/kilbinge/CosmoPMC/pmclib_v1.01.tar.gz
tar zxf pmclib_v1.01.tar.gz
pushd pmclib_v1.01
./waf configure --m64 --prefix=/opt/pkgs/pmclib-1.01
./waf
sudo ./waf install
sudo find /opt/pkgs/pmclib-1.01/include -name "*.h" \
    -exec sed -i \
    -e 's/#include "errorlist\.h"/#include <pmctools\/errorlist.h>/' \
    -e 's/#include "io\.h"/#include <pmctools\/io.h>/' \
    -e 's/#include "mvdens\.h"/#include <pmctools\/mvdens.h>/' \
    -e 's/#include "maths\.h"/#include <pmctools\/maths.h>/' \
    -e 's/#include "maths_base\.h"/#include <pmctools\/maths_base.h>/' \
    {} \;
sudo sed -i \
    -e 's/^double fmin(double/\/\/&/' \
    -e 's/^double fmax(double/\/\/&/' \
    /opt/pkgs/pmclib-1.01/include/pmctools/maths.h
popd
popd
rm -R /tmp/libpmc
\end{commandline}

\textbf{Note}: The \filename{waf} build script used by \package{pmclib} is written in Python2. On
systems that use Python3 as the default interpreter, you will see an error message similar
to the following:
\begin{file}
./waf configure --m64 --prefix=/opt/pkgs/pmclib-1.01
/tmp/src/libpmc/pmclib_v1.01/wscript: error: Traceback (most recent call last):
  File "/tmp/src/libpmc/pmclib_v1.01/.waf3-1.5.17-496be6959d6e0cd406d5f087856c4d79/wafadmin/Utils.py", line 198, in load_module
    exec(compile(code,file_path,'exec'),module.__dict__)
  File "/tmp/src/libpmc/pmclib_v1.01/wscript", line 130
    except Exception,e:
                    ^
SyntaxError: invalid syntax
\end{file}
You can fixed this problem by replacing \texttt{python} with \texttt{python2} in the very first
line of the file \filename{waf} in the top-level \package{pmclib} directory.\\

You can then proceed with the EOS installation as documented in \refsec{inst:EOS}.

\subsection{MacOSX: Installing the Dependencies with \package{Homebrew} and \package{PyPi}}

You can install most of the prerequisite software via \package{Homebrew}.
You will need to make \package{Homebrew} aware of \EOS'
third-party repository as follows:
\begin{commandline}
brew tap eos/eos
\end{commandline}

To install the packages, execute the following commands:
\begin{commandline}
# for the 'System Software'
brew install autoconf automake libtool pkg-config boost yaml-cpp
# for the 'Python Software'
brew install python3 boost-python3
# for the 'Scientific Software'
brew install gsl hdf5 minuit2 pmclib
\end{commandline}

Now you can use the \client{pip3} command\footnote{%
    Due to problems with the \package{Python 3} installation provided by Mac OS
    X, we strongly recommend to use the \package{Homebrew} provided
    \client{pip3} instead via \cli{/usr/local/bin/pip3}~.
}
to install the remaining packages from the \package{PyPi} package index:
\begin{commandline}
pip3 install h5py matplotlib scipy PyYAML
\end{commandline}


You can then proceed with the EOS installation as documented in \refsec{inst:EOS}.

\section{Installing EOS}
\label{sec:inst:EOS}

You can obtain the most recent version of EOS is available from Github
\footnote{\url{http://github.com/eos/eos}}.  To download it for the first time,
clone the repository via
%
\begin{commandline}
git clone -o eos -b master https://github.com/eos/eos.git
\end{commandline}

You must first create all the necessary build scripts via
%
\begin{commandline}
cd eos
./autogen.bash
\end{commandline}
%
You must specify where \EOS will be installed by setting the \envvar{PREFIX}
environment variable.
\begin{enumerate}
    \item If you are administrating the computer on which you install EOS,
    you may choose any prefix. We recommend \directory{/usr/local}:
\begin{commandline}
export PREFIX=/usr/local
\end{commandline}
    We strongly discourage using \directory{/usr}, which avoids conflicts with
    your operating system's package management software.
    \item If you are not administrating the computer, you must install EOS into
    a directory below your home directory or below any other directory for which
    you have the needed permissions. We recommend \directory{\$HOME/.local}:
\begin{commandline}
export PREFIX=$HOME/.local/
\end{commandline}
\end{enumerate}

\subsubsection{Configuration}

Next, you must configure the \EOS build using the \cli{configure} script.
\begin{enumerate}
    \item If you want to use the \EOS Python interface you must pass the \cli{--enable-python} option
    to \cli{configure}. The default is \cli{--disable-python}.

    \item If you want to use the \EOS \gls{PMC} algorithm you must pass the \cli{--enable-pmc} option
    to \cli{configure}. The default is \cli{--disable-pmc}.
    If you installed \package{pmclib} from source to a non-standard location,
    you must specify its installation prefix by passing on
    \cli{--with-pmc=PMC-INSTALLATION-PREFIX}.

    \item If you want to use \package{ROOT}'s internal copy of Minuit2 you must
    pass \cli{--with-minuit2=root} to \cli{configure}.

    \item Otherwise, if you have installed \package{Minuit2} from source to a non-standard location
    you must specify its installation prefix by passing on
    \cli{--with-minuit2=MINUIT2-INSTALLATION-PREFIX}.
\end{enumerate}
%
The recommended configuration is achieved using
\begin{commandline}
./configure \
    --prefix=$PREFIX \
    --enable-python \
    --enable-pmc
\end{commandline}
If the \texttt{configure} script finds any problems with your system, it will complain loudly.\\

\subsubsection{Building and installing}
After successful configuration, build \EOS using
%
\begin{commandline}
make -j all
\end{commandline}
The \cli{-j} option instructs \cli{make} to use all available processors to parallelize
the build process.
%
We strongly recommend testing the build by executing
%
\begin{commandline}
make -j check VERBOSE=1
\end{commandline}
%
within the build directory. Please contact the authors if any
test fails by opening an issue in the official
\EOS Github repository. If all tests pass, install \EOS using
\begin{commandline}
make install # Use 'sudo make install' if you install e.g. to '/usr/local'
             # or a similarly privileged directory
\end{commandline}

If you installed \EOS to a non-standard location (\ie not
\directory{/usr/local}), to use it from the command line you must set up some
environment variable. For \package{BASH}, which is the default Debian/Ubuntu
shell, add the following lines to \filename{\$HOME/.bash\_profile}:
\begin{commandline}
export PATH+=":$PREFIX/bin"
# uncomment the following line for Python support
#export PYTHONPATH+=":$PREFIX/lib/python3.5/site-packages"
\end{commandline}
The last line should be uncommented only if you built \EOS with Python support.
Note that \cli{python3.5} piece must be replaced by the appropriate Python
version with which \EOS was built. You can determine the correct value
using
\begin{commandline}
python3 -c "import sys; print('python{0}.{1}'.format(sys.version_info[0], sys.version_info[1]))"
\end{commandline}

\subsection{MacOS X short cuts}

\subsubsection{For users}
The most recent development version of \EOS (and all of its dependencies) can
also be installed automatically using \package{Homebrew}:
\begin{commandline}
brew tap eos/eos
brew install --HEAD eos
\end{commandline}

\subsubsection{For developers}
If you intend to make changes to the \EOS source code you can still use \package{Homebrew}.
First, you should clone the \EOS repository into your home directory. We suggest
\begin{commandline}
mkdir -p ~/Repositories/eos
cd ~/Repositories/eos
git clone https://github.com/eos/eos.git -b master
# make and commit your changes
\end{commandline}
To setup your local for use with \package{Homebrew} you must ``tap'' the \EOS ``formula'' as usual, and then edit it:
\begin{commandline}
brew tap eos/eos
brew edit eos
\end{commandline}
Change the line starting with \texttt{head} to look as follows:
\begin{file}
  head "/Users/USERNAME/Repositories/eos/.git", :branch => "master", :using => :git
\end{file}
where \texttt{USERNAME} is your username. You can determine it via \cli{whoami}.
Proceed to install the modified \EOS version via \cli{brew install --HEAD eos}.
To reinstall \EOS after further changes to the source code use
\cli{brew reinstall --HEAD eos}.\\
\textbf{Note}: Only changes that have been committed to the \texttt{master} branch of your local clone of the \EOS repository
will be used for the installation.
