\chapter{Extending EOS}
\label{ch:extending}

We discuss the three most common cases of extending \EOS, by adding either
a new parameter, a new observable, or a new measurement.
For extensions that do not fall in these categories please contact the main authors, who will be
happy to discuss your work with you. The most efficient way to contribute new code
or modifications to \EOS is via a pull request in the main Github repository.\footnote{
    \url{https://github.com/eos/eos}
}

\section{How to add a new parameter}
\label{sec:extending:parameter}

Parameters are stored in YAML files below \directory{eos/parameters}, including the default values and range,
as well as metadata and the origin thereof. Each file contains key-to-value maps:
\begin{itemize}
    \item Keys starting and ending with an \texttt{@} sign are considered to be metadata pertaining
    to the entire file.

    \item All other keys declare a new parameter.
\end{itemize}
The values in this map are the definitions of a parameters, and are also structured as maps. Valid
parameter definitions must include all of the following entries:
\begin{description}[labelwidth=.15\textwidth]
    \item[central] the default central value;
    \item[min]     the minimal value for the default range;
    \item[max]     the maximal value for the default range.
\end{description}
Optional entries include:
\begin{description}[labelwidth=.15\textwidth]
    \item[comment] a comment on the purpose;
    \item[latex]   the LaTeX command for the display;
    \item[unit]    the unit in which the parameter is expressed.
\end{description}

\subsection*{Example}

We introduce two new parameters: \parameter{mass::X@Example}, and \parameter{decay-constant::X@Example},
standing in for the mass and decay constant of a hypothetical particle $X$. We add the file
\filename{eos/parameters/x.yaml} with contents:
\fileexample{examples/x-parameters.yaml}

\section{How to add a new observable}
\label{sec:extending:observable}

An Observable is implemented in EOS as follows:
\begin{itemize}
    \item The numerical code is implemented as the method \cpp{double P::o(...) const},
    where \cpp{P} is the underlying class, and the dots stand in for any number
    of \cpp{const double \&} arguments, including zero. The \cpp{class P} must
    inherit from \cpp{class ParameterUser}. It must have one constructor accepting
    a \cpp{class Parameters} and a \cpp{class Options} instance in that order.
    Schematically:
\begin{sourcecode}
#include <eos/utils/parameters.hh>
#include <eos/utils/options.hh>

class P :
    public ParameterUser
{
    P(const Parameters \&, const Options \&};
    ~P();

    double observable_without_kinematics() const;
    double observable_with_one_kinematic_variable(const double &) const;
};
\end{sourcecode}

    \item The method or methods are then associated with their names within
    the file \filename{eos/observable.cc} within the free-standing function
    \cpp{make\_observable\_entries()}. Within the existing map,
    new entries are created through a call to \cpp{make\_observable(...)}.
    For a regular observable the arguments are in order: the name of the observable;
    the pointer to its method; and the tuple of \cpp{std::string}s that names
    the kinematic variables in the order the method expects them. Schematically:
\begin{sourcecode}
# within eos/observable.cc
# [...]
make_observable("P::observable1",
    &P::observable_without_kinematics);
make_observable("P::observable2(variablename)",
    &P::observable_with_one_kinematic_variable,
    std::make_tuple("variablename"));
# [...]
\end{sourcecode}
\end{itemize}

\section{How to add a new constraint}

TODO
