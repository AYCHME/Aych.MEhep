\chapter{Extending EOS}
\label{ch:extending}

We discuss the three most common cases of extending \EOS, by adding either
a new parameter, a new observable, or a new measurement.
For extensions that do not fall in these categories please contact the main authors, who will be
happy to discuss your work with you. The most efficient way to contribute new code
or modifications to \EOS is via a pull request in the main Github repository.\footnote{
    \url{https://github.com/eos/eos}
}

\section{How to add a new parameter}
\label{sec:extending:parameter}

Parameters are stored in YAML files below \directory{eos/parameters}, including the default values and range,
as well as metadata and the origin thereof. Each file contains key-to-value maps:
\begin{itemize}
    \item Keys starting and ending with an \texttt{@} sign are considered to be metadata pertaining
    to the entire file.

    \item All other keys declare a new parameter.
\end{itemize}
The values in this map are the definitions of a parameters, and are also structured as maps. Valid
parameter definitions must include all of the following entries:
\begin{description}[labelwidth=.15\textwidth]
    \item[central] the default central value;
    \item[min]     the minimal value for the default range;
    \item[max]     the maximal value for the default range.
\end{description}
Optional entries include:
\begin{description}[labelwidth=.15\textwidth]
    \item[comment] a comment on the purpose;
    \item[latex]   the LaTeX command for the display;
    \item[unit]    the unit in which the parameter is expressed.
\end{description}

\subsection*{Example}

We introduce two new parameters: \parameter{mass::X@Example}, and \parameter{decay-constant::X@Example},
standing in for the mass and decay constant of a hypothetical particle $X$. We add the file
\filename{eos/parameters/x.yaml} with contents:
\fileexample{examples/x-parameters.yaml}

\section{How to add a new observable}
\label{sec:extending:observable}

An Observable is implemented in EOS as follows:
\begin{itemize}
    \item The numerical code is implemented as the method \cpp{double P::o(...) const},
    where \cpp{P} is the underlying class, and the dots stand in for any number
    of \cpp{const double \&} arguments, including zero. The \cpp{class P} must
    inherit from \cpp{class ParameterUser}. It must have one constructor accepting
    a \cpp{class Parameters} and a \cpp{class Options} instance in that order.
    Schematically:
\begin{sourcecode}
#include <eos/utils/parameters.hh>
#include <eos/utils/options.hh>

class P :
    public ParameterUser
{
    P(const Parameters \&, const Options \&};
    ~P();

    double observable_without_kinematics() const;
    double observable_with_one_kinematic_variable(const double &) const;
};
\end{sourcecode}

    \item The method or methods are then associated with their names within
    the file \filename{eos/observable.cc} within the free-standing function
    \cpp{make\_observable\_entries()}. Within the existing map,
    new entries are created through a call to \cpp{make\_observable(...)}.
    For a regular observable the arguments are in order: the name of the observable;
    the pointer to its method; and the tuple of \cpp{std::string}s that names
    the kinematic variables in the order the method expects them. Schematically:
\begin{sourcecode}
# within eos/observable.cc
# [...]
make_observable("P::observable1",
    &P::observable_without_kinematics);
make_observable("P::observable2(variablename)",
    &P::observable_with_one_kinematic_variable,
    std::make_tuple("variablename"));
# [...]
\end{sourcecode}
\end{itemize}

\section{How to add a new constraint}

Constraints are stored in YAML files below \directory{eos/constraints}. New constraints can be
added to an existing file, or to an entirely new file. Each file contains key-to-value maps
in which the top-level keys declare a new constraint.
The values in this map are the definitions of a constraint, and are also structured as maps. Valid
constraint definitions must at least include a \texttt{type} entry that governs the functional
form of the associated likelihood. \EOS understands the following types of constraints:
\begin{itemize}[labelwidth=.15\textwidth]
    \item \texttt{Amoroso},
    \item \texttt{Gaussian},
    \item \texttt{MultivariateGaussian},
    \item \texttt{MultivariateGaussian(Covariance)}.
\end{itemize}

\subsection{Type \texttt{Amoroso}}

Type \texttt{Amoroso} uses a likelihood arising from the \gls{PDF} of the
Amoroso distribution \cite{Crooks:2010}, a four-parameter distribution. It is
useful for the modeling of upper limits on as-of-yet undiscovered decays at
given probabilities. For example, the upper limits on the decay $B_s \to
\mu^+\mu^-$ by the LHCb experiment prior to the discovery of this decay are
modeled using the \texttt{Amoroso} type of likelihood. The following keys are
required in the description of the constraint:
\begin{description}[labelwidth=.15\textwidth]
    \item[\texttt{observable}] the qualified name for an \texttt{Observable} or a \texttt{Parameter};
    \item[\texttt{kinematics}] the map representation of the kinematic variables pertaining to the observable;
    \item[\texttt{options}] the map representation of the options pertaining to the observable;
    \item[\texttt{physical-limit}] the physical lower bound on the observable;
    \item[\texttt{theta}] the $\theta$ parameter of the Amoroso distribution;
    \item[\texttt{alpha}] the $\alpha$ parameter of the Amoroso distribution;
    \item[\texttt{beta}] the $\beta$ parameter of the Amoroso distribution;
\end{description}
We strongly recommend contacting the \EOS authors before adding your own constraint of type \texttt{Amoroso}.

\subsection{Type \texttt{Gaussian}}

Type \texttt{Gaussian} uses a univariate Gaussian or normally-distributed
likelihood. The following keys are required in the description of the
constraint:
\begin{description}[labelwidth=.15\textwidth]
    \item[\texttt{observable}] the qualified name of an \texttt{Observable} or a \texttt{Parameter};
    \item[\texttt{kinematics}] the map representation of the kinematic variables pertaining to the observable;
    \item[\texttt{options}] the map representation of the options pertaining to the observable;
    \item[\texttt{mean}] the mean of the distribution;
    \item[\texttt{sigma-stat}] the map representation with keys \texttt{hi} and \texttt{lo} for the
    upper and lower statistical uncertainty of the distribution, respectively;
    \item[\texttt{sigma-sys}] the same as \texttt{sigma-stat}, but for the systematic uncertainty;
    \item[\texttt{dof}] the degrees of freedom associated with this observation (should default to $1$).
\end{description}
By default the uncertainties are symmetrized, and statistical and systematical
uncertainties are added in quadrature.

\subsection{Type \texttt{MultivariateGaussian}}

Type \texttt{MultivariateGaussian} uses a multivariate Gaussian likelihood,
parametrized in terms of its mode and correlation matrix. The following keys
are required in the description of the constraint:
\begin{description}[labelwidth=.15\textwidth]
    \item[\texttt{dim}] the dimensionality of the multivariate distribution;
    \item[\texttt{observable}] the list of size \texttt{dim} of qualified names for either \texttt{Observable}s or \texttt{Parameter}s;
    \item[\texttt{kinematics}] the list of size \texttt{dim} of map representations of the kinematic variables pertaining to each observable;
    \item[\texttt{options}] the list of size \texttt{dim} of map representations of the options pertaining to each observable;
    \item[\texttt{mean}] the list of size \texttt{dim} describing the mean of the distribution;
    \item[\texttt{sigma-stat-hi}] the list of size \texttt{dim} for the upper statistical uncertainty of the distribution;
    \item[\texttt{sigma-stat-lo}] the list of size \texttt{dim} for the lower statistical uncertainty of the distribution;
    \item[\texttt{correlations}] the matrix of size \texttt{dim} $\times$ \texttt{dim} for the statistical correlations;
    \item[\texttt{sigma-sys}] the list of size \texttt{dim} for the systematic uncertainty;
    \item[\texttt{dof}] the degrees of freedom associated with this observation (should default to \texttt{dim}).
\end{description}
By default the statistical and systematical uncertainties are added in
quadrature, and the larger of the upper and lower uncertainties are chosen as
the standard deviation. The covariance matrix $\Sigma_{ij}$ is then computed from
the variances $\sigma_i$ and the correlation matrix $\rho_{ij}$ as
\begin{equation}
    \Sigma_{ij} = \sigma_i \sigma_j \rho_{ij}\,.
\end{equation}

\subsection{Type \texttt{MultivariateGaussian(Covariance)}}

Type \texttt{MultivariateGaussian(Covariance)} uses a multivariate Gaussian
likelihood, parametrized in terms of its mode and covariance matrix. The
following keys are required in the description of the constraint:
\begin{description}[labelwidth=.15\textwidth]
    \item[\texttt{dim}] the dimensionality of the multivariate distribution;
    \item[\texttt{observable}] the list of size \texttt{dim} of qualified names for either \texttt{Observable}s or \texttt{Parameter}s;
    \item[\texttt{kinematics}] the list of size \texttt{dim} of map representations of the kinematic variables pertaining to each observable;
    \item[\texttt{options}] the list of size \texttt{dim} of map representations of the options pertaining to each observable;
    \item[\texttt{mean}] the list of size \texttt{dim} describing the mean of the distribution;
    \item[\texttt{covariance}] the matrix of size \texttt{dim} $\times$ \texttt{dim} for the covariance of the distribution;
    \item[\texttt{dof}] the degrees of freedom associated with this observation (should default to \texttt{dim}).
\end{description}

\subsection*{Example}

We introduce a constraint to correlate the new parameters
\parameter{mass::X@Example}, and \parameter{decay-constant::X@Example} from
\refsec{extending:parameter}. The correlation is fixed at $50\%$, and the means
and standard deviations reflect the previous one. We add the file
\filename{eos/constraints/x.yaml} with contents:
\fileexample{examples/x-constraints.yaml}
