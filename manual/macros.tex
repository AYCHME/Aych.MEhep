%% vim : set sts=4 et :

%%
%% Shortcuts and abreviations
%%

%% Physics Shortcuts
\newcommand{\dd}{\mathrm{d}}
\newcommand{\eps}{\varepsilon}
\newcommand{\GeV}{\ensuremath{\mathrm{GeV}}}
\newcommand{\MSbar}{\ensuremath{\overline{\text{MS}}}\xspace}
\newcommand{\op}[1]{\mathcal{O}_{#1}}
\newcommand{\tildeop}[1]{\tilde{\mathcal{O}}_{#1}}
\newcommand{\wilson}[2][{}]{\mathcal{C}_{#2}^{#1}}

%% Math Shortcuts
\newcommand{\para}{\parallel}
\newcommand{\nn}{\nonumber}
\renewcommand{\Re}{\operatorname{Re}}
\renewcommand{\Im}{\operatorname{Im}}

%% Abbreviations
\newcommand{\ie}{i.e.}
\newcommand{\eg}{e.g.}
\newcommand{\cf}{cf.}

%%
%% Formatting
%%

%% create commandline environment
\lstnewenvironment{commandline}[1][\normalsize]{%
\lstset{
    language={},
    basicstyle=\scriptsize\ttfamily{#1},
    showspaces=false,
    showtabs=false,
    breaklines=true,
    showstringspaces=false,
    breakatwhitespace=true,
    prebreak={\ttfamily\symbol{'134}},
    %postbreak={\raisebox{0ex}[0ex][0ex]{\ensuremath{\color{red}\hookrightarrow\space}}},
    xleftmargin=.025\textwidth,
    frame=single,
    backgroundcolor=\color{black!10!white}
}
}{}

%% read in an example commandline from file
\newcommand{\commandlineexample}[2][\normalsize]{%
\lstinputlisting[
    language={},
    basicstyle={\scriptsize\ttfamily{#1}},
    showspaces=false,
    showtabs=false,
    breaklines=true,
    showstringspaces=false,
    breakatwhitespace=true,
    prebreak={\ttfamily\symbol{'134}},
    %postbreak={\raisebox{0ex}[0ex][0ex]{\ensuremath{\color{red}\hookrightarrow\space}}},
    xleftmargin=.025\textwidth,
    frame=single,
    backgroundcolor=\color{black!10!white}
]{#2}
}

%% create file environment
\lstnewenvironment{file}[1][\normalsize]{%
\lstset{
    language={},
    basicstyle={\scriptsize\ttfamily{#1}},
    showspaces=false,
    showtabs=false,
    breaklines=true,
    showstringspaces=false,
    breakatwhitespace=true,
    prebreak={\ttfamily\small\symbol{'134}},
    %postbreak={\raisebox{0ex}[0ex][0ex]{\ensuremath{\color{red}\hookrightarrow\space}}},
    xleftmargin=.025\textwidth,
    frame=single,
    backgroundcolor=\color{blue!25!white}
}
}{}

%% read in an example file
\newcommand{\fileexample}[2][\normalsize]{%
\lstinputlisting[
    language={},
    basicstyle={\scriptsize\ttfamily{#1}},
    showspaces=false,
    showtabs=false,
    breaklines=true,
    showstringspaces=false,
    breakatwhitespace=true,
    prebreak={\ttfamily\symbol{'134}},
    %postbreak={\raisebox{0ex}[0ex][0ex]{\ensuremath{\color{red}\hookrightarrow\space}}},
    xleftmargin=.025\textwidth,
    frame=single,
    backgroundcolor=\color{blue!25!white}
]{#2}
}

%% create source code environment
\lstnewenvironment{sourcecode}[1][\normalsize]{%
\lstset{
    language=c++,
    basicstyle={\scriptsize\ttfamily{#1}},
    showspaces=false,
    showtabs=false,
    breaklines=true,
    showstringspaces=false,
    breakatwhitespace=true,
    prebreak={\ttfamily\small\symbol{'134}},
    %postbreak={\raisebox{0ex}[0ex][0ex]{\ensuremath{\color{red}\hookrightarrow\space}}},
    xleftmargin=.025\textwidth,
    numbers=left,
    numbersep=5pt,
    numberstyle=\color{gray}\tiny,
    frame=single,
    backgroundcolor=\color{orange!25!white}
}
}{}

%% list observables
\newenvironment{observables}{%
\renewcommand{\arraystretch}{1.3}
\begin{center}
\begin{tabular}{l c c}
    \toprule
    Observable name & Mathematical Symbol & Reference\\
    \midrule
}{%
    \bottomrule
\end{tabular}
\end{center}
\renewcommand{\arraystretch}{1}
}
\newcommand{\singleobs}[3]{%
    \texttt{#1} & #2 & #3\\
}
\newcommand{\multiobs}[4]{%
    \texttt{#1} & #2 & \multirow{#4}{*}{#3}\\ \cmidrule{1-2}
}
\newcommand{\multiobsnr}[2]{%
    \texttt{#1} & #2 & \\
}

%% refer to elements of the document
\newcommand{\refapp}[1]{appendix~\ref{app:#1}}
\newcommand{\refch}[1]{chapter~\ref{ch:#1}}
\newcommand{\refeq}[1]{eq.~(\ref{eq:#1})}
\newcommand{\refeqs}[2]{eqs.~(\ref{eq:#1})-(\ref{eq:#2})}
\newcommand{\reffig}[1]{figure~\ref{fig:#1}}
\newcommand{\refsec}[1]{section~\ref{sec:#1}}
\newcommand{\reftab}[1]{table~\ref{tab:#1}}

%% consistent typesetting
% C++
\newcommand{\cpp}[1]{\mbox{\texttt{#1}}}
\newcommand{\class}[1]{\mbox{\texttt{#1}}}
\newcommand{\method}[1]{\mbox{\texttt{#1}}}
% Shell
\newcommand{\cli}[1]{\colorbox{black!10!white}{\texttt{#1}}}
\newcommand{\client}[1]{\colorbox{black!10!white}{\texttt{#1}}}
\newcommand{\envvar}[1]{\colorbox{black!10!white}{\texttt{\$#1}}}
% OS
\newcommand{\directory}[1]{\textcolor{blue!75!white}{\texttt{#1/}}}
\newcommand{\filename}[1]{\textcolor{blue!75!white}{\texttt{#1}}}
\newcommand{\package}[1]{\textcolor{black!50!green}{\textrm{\textbf{#1}}}}
% EOS
\newcommand{\EOS}{\mbox{\textcolor{black!50!white}{\textbf{EOS}}}\xspace}
\newcommand{\observable}[1]{\mbox{\textcolor{orange!75!white}{\texttt{\lstinline{#1}}}}}
\newcommand{\option}[1]{\mbox{\textcolor{orange!75!white}{\texttt{\lstinline{#1}}}}}
\newcommand{\parameter}[1]{\mbox{\textcolor{orange!75!white}{\texttt{\lstinline{#1}}}}}

